\documentclass[8pt,a4paper,compress,handout]{beamer}

\usepackage{/home/siyer/lib/slides}

\title{Control Flow}
\date{}

\begin{document}
\begin{frame}
\vfill
\titlepage
\end{frame}

\begin{frame}
\frametitle{Outline}
\tableofcontents
\end{frame}

\section{.}
\begin{frame}[fragile]
\begin{framed}
\tiny flip.py: Simulate a coin flip by writing 'Heads' or 'Tails' to standard output.
\end{framed}

\begin{lstlisting}[language=Python]
import random
import stdio

if random.randrange(0, 2) == 0:
    stdio.writeln('Heads')
else:
    stdio.writeln('Tails')
\end{lstlisting}

\begin{lstlisting}[language={}]
$ python flip.py 
Tails
$ python flip.py 
Heads
$ python flip.py 
Heads
$ python flip.py 
Tails
$ python flip.py 
Heads
\end{lstlisting}
\end{frame}

\begin{frame}[fragile]
\begin{framed}
\tiny tenhellos.py: Write 10 Hellos to standard output.
\end{framed}

\begin{lstlisting}[language=Python]
import stdio

stdio.writeln('1st Hello')
stdio.writeln('2nd Hello')
stdio.writeln('3rd Hello')
i = 4
while i <= 10:
    stdio.writeln(str(i) + 'th Hello')
    i = i + 1
\end{lstlisting}

\begin{lstlisting}[language={}]
$ python tenhellos.py 
1st Hello
2nd Hello
3rd Hello
4th Hello
5th Hello
6th Hello
7th Hello
8th Hello
9th Hello
10th Hello
\end{lstlisting}
\end{frame}

\begin{frame}[fragile]
\begin{framed}
\tiny powersoftwo.py: Accept positive integer $n$ as a command-line argument. Write to standard output a table showing the first $n$ powers of two.
\end{framed}

\begin{lstlisting}[language=Python]
import stdio
import sys

n = int(sys.argv[1])
power = 1
i = 0
while i <= n:
    stdio.writeln(str(i) + ' ' + str(power))    
    power = 2 * power
    i = i + 1
\end{lstlisting}

\begin{lstlisting}[language={}]
$ python powersoftwo.py 8
0 1
1 2
2 4
3 8
4 16
5 32
6 64
7 128
8 256
\end{lstlisting}
\end{frame}

\begin{frame}[fragile]
\begin{framed}
\tiny divisorpattern.py: Accept integer command-line argument $n$. Write to standard output an $n$-by-$n$ table with an asterisk in row $i$ and column $j$ if either $i$ divides $j$ or $j$ divides $i$.
\end{framed}

\begin{lstlisting}[language=Python]
import stdio
import sys

n = int(sys.argv[1])
for i in range(1, n + 1):
    for j in range(1, n + 1):
        if (i % j == 0) or (j % i == 0):
            stdio.write('* ')
        else:
            stdio.write('  ')
    stdio.writeln(i)
\end{lstlisting}

\begin{lstlisting}[language={}]
$ python divisorpattern.py 10
* * * * * * * * * * 1
* *   *   *   *   * 2
*   *     *     *   3
* *   *       *     4
*       *         * 5
* * *     *         6
*           *       7
* *   *       *     8
*   *           *   9
* *     *         * 10
\end{lstlisting}
\end{frame}

\begin{frame}[fragile]
\begin{framed}
\tiny harmonic.py: Accept integer $n$ as a command-line argument. Write to standard output the $n$th harmonic number $H_n$, computed as $H_n=1+1/2+1/3+\cdots+1/n$. Note that $H_n \approx \ln(n) + 0.57721$ for large $n$.
\end{framed}

\begin{lstlisting}[language=Python]
import stdio
import sys

n = int(sys.argv[1])
sum = 0.0
for i in range(1, n+1):
    sum += 1.0 / i
stdio.writeln(sum)
\end{lstlisting}

\begin{lstlisting}[language={}]
$ python harmonic.py 2
1.5
$ python harmonic.py 10
2.92896825397
$ python harmonic.py 10000
9.78760603604
\end{lstlisting}
\end{frame}

\begin{frame}[fragile]
\begin{framed}
\tiny sqrt.py: Accept a float $c$ as a command-line argument. Write to standard output the square root of $c$ to 15 decimal places of accuracy, calculated using Newton's method.
\end{framed}

\begin{lstlisting}[language=Python]
import stdio
import sys

EPSILON = 1e-15
c = float(sys.argv[1])
t = c
while abs(t - c / t) > (EPSILON * t):
    t = (c / t + t) / 2.0
stdio.writeln(t)
\end{lstlisting}

\begin{lstlisting}[language={}]
$ python sqrt.py 2.0
1.41421356237
$ python sqrt.py 2544545
1595.16300108
\end{lstlisting}
\end{frame}

\begin{frame}[fragile]
\begin{framed}
\tiny binary.py: Accept integer $n$ as a command-line argument. Write the binary representation of $n$ to standard output.
\end{framed}

\begin{lstlisting}[language=Python]
import sys
import stdio

n = int(sys.argv[1])
v = 1
while v <= n / 2:
    v *= 2
while v > 0:
    if n < v:
        stdio.write(0)
    else:
        stdio.write(1)
        n -= v
    v /= 2
stdio.writeln()
\end{lstlisting}

\begin{lstlisting}[language={}]
$ python binary.py 19
10011
$ python binary.py 255
11111111
$ python binary.py 512
1000000000
$ python binary.py 1000000000
111011100110101100101000000000
\end{lstlisting}
\end{frame}

\begin{frame}[fragile]
\begin{framed}
\tiny gambler.py: Accept integer command-line arguments stake, goal, and trialCount. Run trialCount experiments that start with stake dollars and terminate on 0 dollars or goal. Write to standard output the percentage of wins and the average number of bets per experiment.
\end{framed}

\begin{lstlisting}[language=Python]
import random
import stdio
import sys

stake = int(sys.argv[1])
goal = int(sys.argv[2])
trials = int(sys.argv[3])
bets = 0
wins = 0
for t in range(trials):
    cash = stake
    while cash > 0 and cash < goal:
        bets += 1
        if random.randrange(0, 2) == 0:
            cash += 1
        else:
            cash -= 1
    if cash == goal:
        wins += 1
stdio.writeln(str(100 * wins / trials) + '% wins')
stdio.writeln('Avg # bets: ' + str(bets / trials))
\end{lstlisting}

\begin{lstlisting}[language={}]
$ python gambler.py 50 250 100
24% wins
Avg # bets: 12548
$ python gambler.py 500 2500 100
27% wins
Avg # bets: 1124855
\end{lstlisting}
\end{frame}

\begin{frame}[fragile]
\begin{framed}
\tiny factors.py: Accept integer $n$ as a command-line argument. Write to standard output the prime factors of $n$.
\end{framed}

\begin{lstlisting}[language=Python]
import stdio
import sys

n = int(sys.argv[1])
factor = 2
while factor * factor <= n:
    while n % factor == 0:
        n /= factor
        stdio.write(str(factor) + ' ')
    factor += 1
if n > 1:
    stdio.write(n)
stdio.writeln()
\end{lstlisting}

\begin{lstlisting}[language={}]
$ python factors.py 3757208
2 2 2 7 13 13 397
$ python factors.py 287994837222311
17 1739347 9739789
\end{lstlisting}
\end{frame}

\end{document}

\documentclass[8pt,a4paper,compress,handout]{beamer}

\usepackage{/home/siyer/lib/slides}

\title{Control Flow}
\date{}

\begin{document}
\begin{frame}
\begin{flushright}
\tiny \textsc{Programming is usually taught by examples. \\ - Niklaus Wirth}
\end{flushright}
\titlepage
\end{frame}

\begin{frame}
\frametitle{Outline}
\tableofcontents
\end{frame}

\section{If Statement}
\begin{frame}[fragile]
Most computations require different actions for different inputs and one way to express these differences in Python is using the \lstinline{if} statement:

\smallskip

\begin{lstlisting}[language={}]
if <boolean expression>:
    <statement>
    <statement>
    ...
elif <boolean expression>:
    <statement>
    <statement>
    ...
elif <boolean expression>:
    <statement>
    <statement>
    ...
...
else: 
    <statement>
    <statement>
    ...
\end{lstlisting}
\end{frame}

\begin{frame}[fragile]
\begin{framed}
\tiny flip.py: Simulate a coin flip by writing 'Heads' or 'Tails' to standard output.
\end{framed}

\begin{lstlisting}[language=Python]
import random
import stdio

if random.randrange(0, 2) == 0:
    stdio.writeln('Heads')
else:
    stdio.writeln('Tails')
\end{lstlisting}

\begin{lstlisting}[language={}]
$ python flip.py 
Tails
$ python flip.py 
Heads
$ python flip.py 
Heads
$ python flip.py 
Tails
$ python flip.py 
Heads
\end{lstlisting}
\end{frame}

\section{While Statement}
\begin{frame}[fragile]
Many computations are inherently repetitive and the basic Python construct for handling such computations is the \lstinline{while} statement:

\smallskip

\begin{lstlisting}[language={}]
while <boolean expression>:
    <statement>
    <statement>
    ...
\end{lstlisting}
\end{frame}

\begin{frame}[fragile]
\begin{framed}
\tiny tenhellos.py: Write 10 Hellos to standard output.
\end{framed}

\begin{lstlisting}[language=Python]
import stdio

stdio.writeln('1st Hello')
stdio.writeln('2nd Hello')
stdio.writeln('3rd Hello')
i = 4
while i <= 10:
    stdio.writeln(str(i) + 'th Hello')
    i = i + 1
\end{lstlisting}

\begin{lstlisting}[language={}]
$ python tenhellos.py 
1st Hello
2nd Hello
3rd Hello
4th Hello
5th Hello
6th Hello
7th Hello
8th Hello
9th Hello
10th Hello
\end{lstlisting}
\end{frame}

\begin{frame}[fragile]
\begin{framed}
\tiny powersoftwo.py: Accept positive integer $n$ as a command-line argument. Write to standard output a table showing the first $n$ powers of two.
\end{framed}

\begin{lstlisting}[language=Python]
import stdio
import sys

n = int(sys.argv[1])
power = 1
i = 0
while i <= n:
    stdio.writeln(str(i) + ' ' + str(power))    
    power = 2 * power
    i = i + 1
\end{lstlisting}

\begin{lstlisting}[language={}]
$ python powersoftwo.py 8
0 1
1 2
2 4
3 8
4 16
5 32
6 64
7 128
8 256
\end{lstlisting}
\end{frame}

\begin{frame}[fragile]
Modifying a variable is something that we do so often in programming that modern programming languages like Python provide shorthand notations for the purpose.

\bigskip

The most common practice is to abbreviate an assignment statement of the form 

\begin{lstlisting}[language=Python]
i = i + 1
\end{lstlisting}

with the shorthand notation

\begin{lstlisting}[language=Python]
i += 1
\end{lstlisting}

\bigskip

The same notation works for other binary operators, including \lstinline{-}, \lstinline{*}, and \lstinline{/}.

\bigskip

The \emph{scope of a variable} is part of the program where it is defined; statements that follow the definition in the same block (marked by indentation level).
\end{frame}

\section{For Statement}
\begin{frame}[fragile]
An alternative Python construct --- the \lstinline{for} statement --- provides a more compact notation for carrying out repeated computations:

\smallskip

\begin{lstlisting}[language={}]
for <variable> in <iterable object>:
    <statement>
    <statement>
    ...
\end{lstlisting}

\bigskip

The most commonly used iterable objects are the lists containing arithmetic progressions of integers, returned by the built-in function \lstinline{range()}:

\begin{lstlisting}[language={}]
range(start, stop[, step])
range(stop)
\end{lstlisting}

The first form returns a list of integers, starting at \lstinline{start}, ending just before \lstinline{stop}, and in increments (or decrements) given by the optional \lstinline{step} argument, which defaults to 1. In the second form, \lstinline{start} and \lstinline{step} default to 0 and 1 respectively.
\end{frame}

\begin{frame}[fragile]
The \lstinline{tenhellos.py} program can be written using \lstinline{for} statement as follows:

\begin{lstlisting}[language=Python]
import stdio

stdio.writeln('1st Hello')
stdio.writeln('2nd Hello')
stdio.writeln('3rd Hello')
for i in range(4, 11):
    stdio.writeln(str(i) + 'th Hello')
\end{lstlisting}

\bigskip

Strings are iterable objects. For example, the following code iterates over the characters of string the \lstinline{'AGCT'}:
\begin{lstlisting}[language=Python]
import stdio

for c in 'AGCT':
    stdio.writeln(c)
\end{lstlisting}
\end{frame}

\section{Nesting}
\begin{frame}[fragile]
The \lstinline{if}, \lstinline{while}, and \lstinline{for} statements, collectively called \emph{control-flow statements}, have the same status as assignment statements or any other statements in Python.

\bigskip

As a result, we can use a control-flow statement whenever a statement is called for. 

\bigskip

In particular, we can nest one or more of the control-flow statements in the body of another. 
\end{frame}

\begin{frame}[fragile]
\begin{framed}
\tiny divisorpattern.py: Accept integer command-line argument $n$. Write to standard output an $n$-by-$n$ table with an asterisk in row $i$ and column $j$ if either $i$ divides $j$ or $j$ divides $i$.
\end{framed}

\begin{lstlisting}[language=Python]
import stdio
import sys

n = int(sys.argv[1])
for i in range(1, n + 1):
    for j in range(1, n + 1):
        if (i % j == 0) or (j % i == 0):
            stdio.write('* ')
        else:
            stdio.write('  ')
    stdio.writeln(i)
\end{lstlisting}

\begin{lstlisting}[language={}]
$ python divisorpattern.py 10
* * * * * * * * * * 1
* *   *   *   *   * 2
*   *     *     *   3
* *   *       *     4
*       *         * 5
* * *     *         6
*           *       7
* *   *       *     8
*   *           *   9
* *     *         * 10
\end{lstlisting}
\end{frame}

\section{Applications}
\begin{frame}[fragile]
\begin{enumerate}
\item Finite sum (\lstinline{harmonic.py}).

\item Computing the square root using Newton's method (\lstinline{sqrt.py}).

\item Number conversion (\lstinline{binary.py}).

\item Monte Carlo simulation (\lstinline{gambler.py}).

\item Factoring (\lstinline{factors.py}).
\end{enumerate}
\end{frame}

\begin{frame}[fragile]
\begin{framed}
\tiny harmonic.py: Accept integer $n$ as a command-line argument. Write to standard output the $n$th harmonic number $H_n$, computed as $H_n=1+1/2+1/3+\cdots+1/n$. Note that $H_n \approx \ln(n) + 0.57721$ for large $n$.
\end{framed}

\begin{lstlisting}[language=Python]
import stdio
import sys

n = int(sys.argv[1])
total = 0.0
for i in range(1, n + 1):
    total += 1.0 / i
stdio.writeln(total)
\end{lstlisting}

\begin{lstlisting}[language={}]
$ python harmonic.py 2
1.5
$ python harmonic.py 10
2.92896825397
$ python harmonic.py 10000
9.78760603604
\end{lstlisting}
\end{frame}

\begin{frame}[fragile]
\begin{framed}
\tiny sqrt.py: Accept a float $c$ as a command-line argument. Write to standard output the square root of $c$ to 15 decimal places of accuracy, calculated using Newton's method.
\end{framed}

\begin{lstlisting}[language=Python]
import stdio
import sys

EPSILON = 1e-15
c = float(sys.argv[1])
t = c
while abs(t - c / t) > (EPSILON * t):
    t = (c / t + t) / 2.0
stdio.writeln(t)
\end{lstlisting}

\begin{lstlisting}[language={}]
$ python sqrt.py 2.0
1.41421356237
$ python sqrt.py 2544545
1595.16300108
\end{lstlisting}
\end{frame}

\begin{frame}[fragile]
\begin{framed}
\tiny binary.py: Accept integer $n$ as a command-line argument. Write the binary representation of $n$ to standard output.
\end{framed}

\begin{lstlisting}[language=Python]
import sys
import stdio

n = int(sys.argv[1])
v = 1
while v <= n // 2:
    v *= 2
while v > 0:
    if n < v:
        stdio.write(0)
    else:
        stdio.write(1)
        n -= v
    v //= 2
stdio.writeln()
\end{lstlisting}

\begin{lstlisting}[language={}]
$ python binary.py 19
10011
$ python binary.py 255
11111111
$ python binary.py 512
1000000000
$ python binary.py 1000000000
111011100110101100101000000000
\end{lstlisting}
\end{frame}

\begin{frame}[fragile]
\begin{framed}
\tiny gambler.py: Accept integer command-line arguments $stake$, $goal$, and $trials$. Run $trials$ experiments that start with $stake$ dollars and terminate on 0 dollars or $goal$. Write to standard output the percentage of wins and the average number of bets per experiment, which can be calculated as $100 \times stake / goal$ and $stake \times (goal - stake)$, respectively.
\end{framed}

\begin{lstlisting}[language=Python]
import random
import stdio
import sys

stake = int(sys.argv[1])
goal = int(sys.argv[2])
trials = int(sys.argv[3])
bets = 0
wins = 0
for t in range(trials):
    cash = stake
    while cash > 0 and cash < goal:
        bets += 1
        if random.randrange(0, 2) == 0:
            cash += 1
        else:
            cash -= 1
    if cash == goal:
        wins += 1
stdio.writeln(str(100 * wins // trials) + '% wins')
stdio.writeln('Avg # bets: ' + str(bets // trials))
\end{lstlisting}

\begin{lstlisting}[language={}]
$ python gambler.py 50 250 100
24% wins
Avg # bets: 12548
$ python gambler.py 500 2500 100
27% wins
Avg # bets: 1124855
\end{lstlisting}
\end{frame}

\begin{frame}[fragile]
\begin{framed}
\tiny factors.py: Accept integer $n$ as a command-line argument. Write to standard output the prime factors of $n$.
\end{framed}

\begin{lstlisting}[language=Python]
import stdio
import sys

n = int(sys.argv[1])
factor = 2
while factor * factor <= n:
    while n % factor == 0:
        n //= factor
        stdio.write(str(factor) + ' ')
    factor += 1
if n > 1:
    stdio.write(n)
stdio.writeln()
\end{lstlisting}

\begin{lstlisting}[language={}]
$ python factors.py 3757208
2 2 2 7 13 13 397
$ python factors.py 287994837222311
17 1739347 9739789
\end{lstlisting}
\end{frame}

\section{Other Conditional and Loop Constructs}
\begin{frame}[fragile]
The \emph{conditional expression} supports an alternate form of an \lstinline$if-else$ statement:
 
\begin{lstlisting}[language={}]
<expression> if <boolean expression> else <expression>
\end{lstlisting}

For example, the following code assigns \lstinline{'Heads'} or \lstinline{'Tails'} to the variable \lstinline{flip}, each with probability $1/2$:

\begin{lstlisting}[language=Java]
flip = 'Heads' if random.random() < 0.5 else 'Tails'
\end{lstlisting}

\bigskip

The \lstinline{break} statement immediately exits a loop without letting it to run to completion. For example, the following code tests and prints if a number $N$ is prime or not:
\begin{lstlisting}[language=Python]
i = 2
while i <= N / i:
    if N % i == 0:
        break
    i += 1
if i > N / i:
    stdio.writeln(str(N) + ' is prime')
else 
    stdio.writeln(str(N) + ' is not prime')
\end{lstlisting}

\bigskip

The \lstinline{continue} statement skips to next iteration of a loop.
\end{frame}

\section{Infinite Loops}
\begin{frame}[fragile]
What if the loop continuation condition is always true? 

\bigskip 
 
\begin{framed}
\tiny badhellos.py: Loops forever, since the loop continuation condition is always true.
\end{framed}

\begin{lstlisting}[language=Python]
import stdio

stdio.writeln('1st Hello')
stdio.writeln('2nd Hello')
stdio.writeln('3rd Hello')
i = 4
while i > 3:
    stdio.writeln(str(i) + 'th Hello')
    i = i + 1
\end{lstlisting}

\begin{lstlisting}[language={}]
$ python badhellos.py 
1st Hello
2nd Hello
3rd Hello
4th Hello
5th Hello
6th Hello
7th Hello
8th Hello
9th Hello
10th Hello
11th Hello
...
\end{lstlisting}
\end{frame}

\end{document}

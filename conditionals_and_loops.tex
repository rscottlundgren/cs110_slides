\documentclass[8pt,a4paper,compress]{beamer}
%\documentclass[8pt,a4paper,compress,handout]{beamer}

\usepackage{amsmath, amssymb, amsthm}
\usepackage{enumerate}
\usepackage{framed}
\usepackage{listings}
\usepackage{tikz}

\usecolortheme{dove}
\useinnertheme{circles}
\beamertemplatenavigationsymbolsempty
\setbeamertemplate{headline}
{
  \leavevmode%
  \hbox{%
  \begin{beamercolorbox}[wd=\paperwidth,ht=6ex]{secsubsec}%
    \raggedright
    \hspace*{1.5em}%
    \normalsize
    \ifx\insertsection\empty\else
      \textbf{\insertsection\text{ }}%
      \ifx\insertsubsection\empty\else
        \textbf{$\bullet$\text{ }\insertsubsection}%
      \fi
    \fi
    \hspace*{2em}%
  \end{beamercolorbox}%
  }%
}
\setbeamerfont{frametitle}{size=\normalsize}
\setbeamertemplate{mini frames}{}
\setbeamertemplate{footline}[page number]

\definecolor{lightgray}{RGB}{240,240,240}
\definecolor{darkgreen}{RGB}{51,102,0}

\title{1.3 Conditionals and Loops}
\date{}

\lstset{
  backgroundcolor=\color{lightgray},
  basicstyle=\footnotesize\ttfamily,
  showstringspaces=false,
  commentstyle=\color{darkgreen},
  keywordstyle=\color{blue},
  stringstyle=\color{orange},
}

\begin{document}

\begin{frame}
\vfill
\titlepage
\end{frame}

\begin{frame}
\frametitle{Outline}
\tableofcontents
\end{frame}

\section{If Statement}
\begin{frame}[fragile]
\pause

Control-flow (statement sequencing in a program) statement used when different actions are required for different inputs.

\begin{lstlisting}[language=Java]
if (<boolean expression>) {
    <statements>
}
else if (<boolean expression>) {
    <statements>
}
...
else {
    <statements>
}
\end{lstlisting}

\pause
\smallskip

\textbf{Program 1.3.1} Flipping a fair coin

\begin{lstlisting}[language=Java]
public class Flip {
    public static void main(String[] args) { 
        if (Math.random() < 0.5) { 
            System.out.println("Heads");
        }
        else {                     
            System.out.println("Tails");
        }
    }
}
\end{lstlisting}

\pause

\begin{lstlisting}[language=bash]
$ java Flip 
Heads
$ java Flip 
Heads
$ java Flip 
Tails
\end{lstlisting}
\end{frame}

\section{While Statement}
\begin{frame}[fragile]
\pause
Control-flow statement for handling repetitive computations.
\begin{lstlisting}[language=Java]
while (<boolean expression>) {
    <statements>
}
\end{lstlisting}

\pause
\smallskip

\textbf{Program 1.3.2} Your first while loop

\begin{lstlisting}[language=Java]
public class TenHellos { 
   public static void main(String[] args) {
      System.out.println("1st Hello");
      System.out.println("2nd Hello");
      System.out.println("3rd Hello");
      int i = 4;
      while (i <= 10) {
         System.out.println(i + "th Hello");
         i = i + 1;
      }
   }
}
\end{lstlisting}

\pause

\begin{lstlisting}[language=bash]
$ java TenHellos 
1st Hello
2nd Hello
3rd Hello
4th Hello
5th Hello
6th Hello
7th Hello
8th Hello
9th Hello
10th Hello
\end{lstlisting}
\end{frame}

\begin{frame}[fragile]
\pause

\textbf{Program 1.3.3} Computing powers of two

\begin{lstlisting}[language=Java]
public class PowersOfTwo {
    public static void main(String[] args) {
        int N = Integer.parseInt(args[0]);
        int i = 0;
        int powerOfTwo = 1;
        while (i <= N) {
            System.out.println(i + " " + powerOfTwo);
            powerOfTwo = 2 * powerOfTwo; 
            i = i + 1;
        }
    }
}
\end{lstlisting}

\pause

\begin{lstlisting}[language=bash]
$ java PowersOfTwo 5
0 1
1 2
2 4
3 8
4 16
5 32
\end{lstlisting}
\end{frame}


\section{For Statement}
\begin{frame}[fragile]
\pause
Alternate (more compact) notation for carrying out repeated computations.
\begin{lstlisting}[language=Java]
for (<initialize>; <boolean expression>; <increment>) {
    <statements>
}
\end{lstlisting}
which is roughly equivalent to 
\begin{lstlisting}[language=Java]
<initialize>;
while (<boolean expression>) {
    <statements>
    <increment>;
}
\end{lstlisting}

\pause
\smallskip

\textbf{Compound Assignment Idiom} Shorthand notation for modifying the value of a variable. Eg, the following statements are equivalent
\begin{lstlisting}[language=Java]
i = i + 1;
i++;
i += 1;
\end{lstlisting}
and so are
\begin{lstlisting}[language=Java]
v *= 2;
v = 2 * v;
v += v;
\end{lstlisting}

\pause
\smallskip

\textbf{Scope of a Variable} Part of the program where it is defined;  statements that follow the declaration in the same block (marked by \lstinline${...}$) as the declaration.

\end{frame}

\section{Nesting}
\begin{frame}[fragile]
\pause

The \lstinline$if$, \lstinline$while$, and \lstinline$for$ statements have the same status as any other statement in Java, ie, we can use them wherever a statement is called for.

\pause
\smallskip

\textbf{Program 1.3.4} Your first nested loops

\begin{lstlisting}[language=Java]
public class DivisorPattern {
    public static void main(String[] args) { 
        int N = Integer.parseInt(args[0]);
        for (int i = 1; i <= N; i++) {
            for (int j = 1; j <= N; j++) {
                if (i % j == 0 || j % i == 0) {
                    System.out.print("* "); 
                }
                else { 
                    System.out.print("  ");
                } 
            }
            System.out.println(i);
        }
    }
}
\end{lstlisting}

\pause

\begin{lstlisting}[language=bash]
$ java DivisorPattern 10
* * * * * * * * * * 1
* *   *   *   *   * 2
*   *     *     *   3
* *   *       *     4
*       *         * 5
* * *     *         6
*           *       7
* *   *       *     8
*   *           *   9
* *     *         * 10
\end{lstlisting}
\end{frame}

\section{Applications}

\begin{frame}[fragile]
\pause

\textbf{Program 1.3.5} Harmonic numbers

\begin{framed}
\tiny Evaluates $H_N=1+\frac{1}{2}+\frac{1}{3}+\dots+\frac{1}{N}\approx\ln(N)+0.57721$ for $N\gg 1$.
\end{framed}

\begin{lstlisting}[language=Java]
public class Harmonic { 
    public static void main(String[] args) { 
        int N = Integer.parseInt(args[0]);
        double sum = 0.0;
        for (int i = 1; i <= N; i++) {
            sum += 1.0 / i;
        }
        System.out.println(sum);
    }
}
\end{lstlisting}

\pause

\begin{lstlisting}[language=bash]
$ java Harmonic 2
1.5
$ java Harmonic 10
2.9289682539682538
$ java Harmonic 10000
9.787606036044348
\end{lstlisting}
\end{frame}

\begin{frame}[fragile]
\pause

\textbf{Program 1.3.6} Newton's method

\begin{framed}
\tiny Uses Newton's method of successive approximation to compute $\sqrt{c}$, ie, a value of $x$ such that $f(x)=x^2-c=0$.
\end{framed}

\begin{lstlisting}[language=Java]
public class Sqrt { 
    public static void main(String[] args) { 
        double c = Double.parseDouble(args[0]);
        double epsilon = 1e-15;
        double t = c; 
        while (Math.abs(t - c/t) > epsilon * t) {
            t = (c / t + t) / 2.0;
        }
        System.out.println(t);
    }
}
\end{lstlisting}

\pause

\begin{lstlisting}[language=bash]
$ java Sqrt 2
1.414213562373095
$ java Sqrt 2544545
1595.1630010754388
\end{lstlisting}
\end{frame}

\begin{frame}[fragile]
\pause

\textbf{Program 1.3.7} Converting to binary

\begin{framed}
\tiny Prints the binary representation of the input number $n$. Eg, $19_{10}=10011_{2}$.
\end{framed}

\begin{lstlisting}[language=Java]
public class Binary { 
    public static void main(String[] args) { 
        int n = Integer.parseInt(args[0]);
        int v = 1;
        while (v <= n / 2) {
            v = v * 2;
        }
        while (v > 0) {
            if (n < v) { 
                System.out.print(0);
            }
            else {
                System.out.print(1);
                n = n - v;
            }
            v = v / 2;
        }
        System.out.println();
    }
}
\end{lstlisting}

\pause

\begin{lstlisting}[language=bash]
$ java Binary 19
10011
$ java Binary 100000000
101111101011110000100000000
\end{lstlisting}
\end{frame}

\begin{frame}[fragile]
\pause

\textbf{Program 1.3.8} Gambler's ruin simulation

\begin{framed}
\tiny A gambler, starting from an initial stake, makes fair \$1 bets, and decides to walk away after reaching a goal. This program addresses two questions: What is the probability $W$ that (s)he will win? What is the average number $B$ of bets (s)he needs to make before winning/losing the game? Analytically, $W=\text{stake}/\text{goal}$ and $B=\text{stake}(\text{goal}-\text{stake})$.
\end{framed}

\begin{lstlisting}[language=Java]
public class Gambler { 
    public static void main(String[] args) {
        int stake = Integer.parseInt(args[0]);
        int goal  = Integer.parseInt(args[1]); 
        int T     = Integer.parseInt(args[2]); 
        int bets = 0; 
        int wins = 0; 
        for (int t = 0; t < T; t++) {
            int cash = stake;
            while (cash > 0 && cash < goal) {
                bets++;
                if (Math.random() < 0.5) {
                    cash++;
                }
                else {                     
                    cash--;
                }
            }
            if (cash == goal) {
                wins++;
            }
        }
        System.out.println(wins + " wins of " + T);
        System.out.println("Percent of games won = " + 100.0 * wins / T);
        System.out.println("Avg # bets           = " + 1.0 * bets / T);
    }
}
\end{lstlisting}
\end{frame}

\begin{frame}[fragile]
\pause

\begin{lstlisting}[language=Bash]
$ java Gambler 10 20 1000
462 wins of 1000
Percent of games won = 46.2
Avg # bets           = 96.638
$ java Gambler 50 250 100
21 wins of 100
Percent of games won = 21.0
Avg # bets           = 11551.3
$ java Gambler 500 2500 100
24 wins of 100
Percent of games won = 24.0
Avg # bets           = 972914.28
\end{lstlisting}
\end{frame}

\begin{frame}[fragile]
\pause

\textbf{Program 1.3.9} Factoring integers

\begin{framed}
\tiny Prints the prime factorization of the input number $n$, ie, the multiset of primes whose product is $n$. Eg, $3757208 = 2 \times 2 \times 2 \times 7 \times 13 \times 13 \times 397$.
\end{framed}

\begin{lstlisting}[language=Java]
public class Factors {
    public static void main(String[] args) { 
        long n = Long.parseLong(args[0]);
        System.out.print("The prime factorization of " + n + " is: ");
        for (long i = 2; i * i <= n; i++) { 
            while (n % i == 0) {
                System.out.print(i + " "); 
                n = n / i;
            }
        }
        if (n > 1) { 
            System.out.println(n);
        }
        else {       
            System.out.println();
        }
    }
}
\end{lstlisting}

\pause

\begin{lstlisting}[language=bash]
$ java Factors 3757208
The prime factorization of 3757208 is: 2 2 2 7 13 13 397
$ java Factors 287994837222311
The prime factorization of 287994837222311 is: 17 1739347 9739789
\end{lstlisting}
\end{frame}

\section{Other Conditional and Loop Constructs}
\begin{frame}[fragile]
\pause

\textbf{Conditional Operator} Alternate form for an \lstinline$if-else$ statement. Eg, 

\begin{lstlisting}[language=Java]
String flip = Math.random() < 0.5 ? "Heads" : "Tails";
\end{lstlisting}

\pause
\smallskip

\textbf{Switch Statement} Alternate form for an \lstinline$if$ statement. Eg, 

\begin{lstlisting}[language=Java]
switch (day) {
    case 0:
        System.out.println("Sunday");
        break;
    case 1:
        System.out.println("Monday");
        break;
    ...
    case 6:
        System.out.println("Saturday");
        break;
    default:
        System.out.println("Error!");
}
\end{lstlisting}

\end{frame}

\begin{frame}[fragile]
\pause

\textbf{Break Statement} Immediately exits a loop without letting it to run to completion. Eg,
\begin{lstlisting}[language=Java]
int i;
for (i = 2; i <= N/i; i++) {
    if (N % i == 0) {
        break;
    }
}
if (i > N / i) {
    System.out.println(N + " is prime");
}
else {
    System.out.println(N + " is not prime");
}
\end{lstlisting}

\pause
\smallskip

\textbf{Continue Statement} Skips to next iteration of a loop.

\pause
\smallskip

\textbf{Do-while Statement} Another way to write a loop.

\begin{lstlisting}[language=Java]
do {
    <statements>
} while (<boolean expression>);
\end{lstlisting}
which is equivalent to 
\begin{lstlisting}[language=Java]
while (<boolean expression>){
    <statements>
}
\end{lstlisting}
except that the first test of the condition is omitted. Eg,
\begin{lstlisting}[language=Java]
do {
    x = 2 * Math.random() - 1.0;
    y = 2 * Math.random() - 1.0;     
} while (Math.sqrt(x * x + y * y) > 1.0);
\end{lstlisting}
samples a point $(x, y)$ from a unit disk centered at the origin.
\end{frame}

\section{Infinite Loops}
\begin{frame}[fragile]
\pause

What if the loop continuation condition is always true? Eg, 
\begin{lstlisting}[language=Java]
public class BadHellos {
    public static void main(String[] args) {
        System.out.println("1st Hello");
        System.out.println("2nd Hello");
        System.out.println("3rd Hello");
        int i = 4;
        while (i > 3) {
            System.out.println(i + "th Hello");
            i = i + 1;
        }
    }
}
\end{lstlisting}

\pause

\begin{lstlisting}[language=bash]
$ java BadHellos
1st Hello
2nd Hello
3rd Hello
4th Hello
5th Hello
6th Hello
7th Hello
...
\end{lstlisting}

\end{frame}

\end{document}

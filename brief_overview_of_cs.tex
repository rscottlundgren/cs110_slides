\documentclass[8pt,a4paper,compress]{beamer}
%\documentclass[8pt,a4paper,compress,handout]{beamer}

\usepackage{amsmath, amssymb, amsthm}
\usepackage{enumerate}
\usepackage{listings}
\usepackage{tikz}
\usepackage{wrapfig}

\usecolortheme{dove}
\useinnertheme{circles}
\beamertemplatenavigationsymbolsempty
\setbeamertemplate{headline}
{
  \leavevmode%
  \hbox{%
  \begin{beamercolorbox}[wd=\paperwidth,ht=6ex]{secsubsec}%
    \raggedright
    \hspace*{1.5em}%
    \normalsize
    \ifx\insertsection\empty\else
      \textbf{\insertsection\text{ }}%
      \ifx\insertsubsection\empty\else
        \textbf{$\bullet$\text{ }\insertsubsection}%
      \fi
    \fi
    \hspace*{2em}%
  \end{beamercolorbox}%
  }%
}
\setbeamerfont{frametitle}{size=\normalsize}
\setbeamertemplate{mini frames}{}
\setbeamertemplate{footline}[page number]

\definecolor{lightgray}{RGB}{240,240,240}
\definecolor{darkgreen}{RGB}{51,102,0}

\title{Brief Overview of Computer Science}
\date{}

\lstset{
  backgroundcolor=\color{lightgray},
  basicstyle=\footnotesize\ttfamily,
  showstringspaces=false,
  commentstyle=\color{darkgreen},
  keywordstyle=\color{blue},
  stringstyle=\color{orange},
}

\begin{document}
\begin{frame}
\vfill
\titlepage
\end{frame}

\begin{frame}
\frametitle{Outline}
\tableofcontents
\end{frame}

\section{What is Computer Science?}
\begin{frame}[fragile]
\begin{flushright}
\tiny \textsc{It is unworthy of excellent men to lose hours like slaves in the labor of calculation which could safely be relegated to anyone else if machines were used. - \href{http://en.wikipedia.org/wiki/Gottfried_Wilhelm_Leibniz}{Gottfried Wilhelm Leibniz}}

\bigskip

\tiny \textsc{Computers are useless. They can only give you answers. - \href{http://en.wikipedia.org/wiki/Pablo_Picasso}{Pablo Picasso}}

\bigskip

\textsc{The computer is incredibly fast, accurate, and stupid. Man is incredibly slow, inaccurate, and brilliant. The marriage of the two is a force beyond calculation. - \href{http://en.wikipedia.org/wiki/Leo_Cherne}{Leo Cherne}}
\end{flushright}

\pause
\textbf{Computer} A general-purpose device that can be programmed to carry out a set of arithmetic or logical operations automatically.

\pause
\smallskip

\textbf{Computer Science} Study of automating algorithmic processes that scale.

\pause
\smallskip

\textbf{Computer Scientist} One who specializes in the theory of computation and the design of computational systems.
\end{frame}

\section{History of Computing}
\begin{frame}[fragile]
\pause

\textbf{1642} Blaise Pascal designed and constructed the first working mechanical calculator, called Pascal's Calculator.

\pause
\smallskip

\textbf{1673} Gottfried Wilhelm Leibniz demonstrated a digital mechanical calculator, called the Stepped Reckoner. He also documented the binary number system and thus is considered the first computer scientist and information theorist.

\pause
\smallskip

\textbf{1820} Thomas de Colmar launched the mechanical calculator industry and released his simplified arithmometer, the first calculating machine strong and reliable enough for use in an office environment.

\pause
\smallskip

\textbf{1822} Charles Babbage started the design of the first automatic mechanical calculator, his Difference Engine.

\pause
\smallskip

\textbf{1834} Charles Babbage started developing the first programmable mechanical calculator, his Analytical Engine. A crucial step was the adoption of a punched-card system derived from the Jacquard loom making it infinitely programmable.

\pause
\smallskip

\textbf{1843} Ada Lovelace wrote an algorithm to compute the Bernoulli numbers, which is considered to be the first computer program.

\pause
\smallskip

\textbf{1885} Herman Hollerith invented the tabulator, which used punched cards to process statistical information.

\pause
\smallskip

\textbf{1937} Howard Aiken convinced IBM to develop his giant programmable calculator, the ASCC/Harvard Mark I.

\pause
\smallskip

\textbf{1962} The first computer science degree program was formed in the United States at Purdue University. 
\end{frame}

\begin{frame}[fragile]
\pause

Computer science, along with Electronics, is a founding science of the current epoch of human history called the Information Age. Some if its major achievements include:
\begin{itemize}
\pause
\item Start of the ``digital revolution'', including the Information Age and the Internet.
\pause
\item Formal definition of computation and computability, and proof that there are computationally unsolvable and intractable problems.
\pause
\item Concept of a programming language as a tool for the precise expression of methodological information at various levels of abstraction.
\pause
\item Breaking of the Enigma code leading to the Allied victory in World War II.
\pause
\item Scientific computing enabled evaluation of processes of great complexity such as computational fluid dynamics, and experimentation entirely by software (aka in-silico experimentation). It also enabled advanced study of the mind, mapping of the human genome, and distributed computing projects such as Folding@Home.
\pause
\item Computer graphics and computer-generated imagery have become ubiquitous in modern entertainment.
\pause
\item Artificial intelligence (AI) is becoming increasingly important. There are many applications of the AI, some of which are domestic, such as robotic vacuum cleaners.
\end{itemize}
\end{frame}

\section{Areas of Computer Science}
\begin{frame}[fragile]
\pause
\textbf{Theoretical Computer Science} Encompasses topics that focus on the more abstract, logical, and mathematical aspects of computing.
\begin{itemize}
\item Theory of Computation (CS420). 
\item Information and Coding Theory. 
\item Algorithms and Data Structures (CS210, CS310).
\item Programming Language Theory (CS450, CS451).
\item Formal Methods.
\end{itemize}

\pause
\textbf{Applied Computer Science} Identifies computer science concepts that can be used directly for solving real-world problems.
\begin{itemize}
\item Artificial Intelligence (CS470).
\item Computer Architecture and Engineering (CS341, CS444).
\item Computer Performance Analysis.
\item Computer Graphics and Visualization (CS447, CS460).
\item Computer Security and Cryptography (CS449).
\item Computation Science.
\item Computer Networks (CS446).
\item Concurrent, Parallel, and Distributed Systems.
\item Databases (CS430, CS436, CS437).
\item Health Informatics.
\item Information Science (CS438).
\item Software Engineering (CS410).
\end{itemize}
\end{frame}

\section{The Great Insights of Computer Science}
\begin{frame}[fragile]
\pause
\textbf{First Great Insight} (Leibniz, Boole, Turing, Shannon, Morse) All the information about any computable problem can be represented using only 0 and 1 (or any other bistable pair).

\pause
\smallskip

\textbf{Second Great Insight} (Turing) Every algorithm can be expressed in a computer language consisting of only five basic instructions:
\begin{itemize}
\item \lstinline$move left$ one location; 
\item \lstinline$move right$ one location; 
\item \lstinline$read symbol$ at current location; 
\item \lstinline$print 0$ at current location; and 
\item \lstinline$print 1$ at current location.
\end{itemize}

\pause
\smallskip

\textbf{Third Great Insight} (B\"{o}hm, Jacopini) Only three rules are needed to combine any set of basic instructions into more complex ones: 
\begin{itemize}
\item Sequence: first do this; then do that; 
\item Selection: \lstinline$IF$ such and such is the case \lstinline$THEN$ do this \lstinline$ELSE$ do that; and 
\item Repetition: \lstinline$WHILE$ such and such is the case \lstinline$DO$ this.
\end{itemize}
\end{frame}

\end{document}

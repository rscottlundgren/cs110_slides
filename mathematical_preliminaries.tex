\documentclass[8pt,a4paper,compress]{beamer}

\usepackage{/home/siyer/lib/slides}

\title{Mathematical Preliminaries}
\date{}

\newlength{\myMheight}
\settoheight{\myMheight}{M}

\begin{document}
\begin{frame}
\vfill
\titlepage
\end{frame}

\section{Numbers}
\begin{frame}[fragile]
\pause\transdissolve

An integer is a number that can be written without a fractional component

\pause\transdissolve\bigskip

Examples of integers: -128, 0, 42

\pause\transdissolve\bigskip

An integer is even if it is divisible by 2, and odd otherwise

\pause\transdissolve\bigskip

The set of all integers is denoted as $\mathbb{Z}$

\pause\transdissolve\bigskip

The set $\mathbb{Z}$ includes zero (0), the natural numbers $\mathbb{N}$ ($1, 2, 3, \dots$), and their additive inverses ($-1, -2, -3, \dots$)
\end{frame}

\begin{frame}[fragile]
\pause\transdissolve

A rational number is a number that can be expressed as the quotient or fraction $p/q$ of two integers, a numerator $p$ and a non-zero denominator $q$

\pause\transdissolve\bigskip

Examples of rational numbers: 42, 2.5, 1.333...

\pause\transdissolve\bigskip

The set of all rational numbers is denoted as $\mathbb{Q}$
\end{frame}

\begin{frame}[fragile]
\pause\transdissolve

A real number is a value of a continuous quantity such as distance or time

\pause\transdissolve\bigskip

Examples of real numbers: -7, 0, 42, 1.333..., $\sqrt{2}$ (1.41421356...), $\pi$ (3.14159265...)

\pause\transdissolve\bigskip

The set of all real numbers is denoted as $\mathbb{R}$

\pause\transdissolve\bigskip

Hierarchy of number systems ($\subset$ means proper subset): $\mathbb{N} \subset \mathbb{Z} \subset \mathbb{Q} \subset \mathbb{R}$

\pause\transdissolve\bigskip

A real number that is not rational is called an irrational number

\pause\transdissolve\bigskip

Examples of irrational numbers: $\sqrt{2}$ (1.41421356...), $\pi$ (3.14159265...)
\end{frame}

\begin{frame}[fragile]
\pause\transdissolve

A prime number is a natural number greater than 1 that cannot be formed by multiplying two smaller natural numbers

\pause\transdissolve\bigskip

Examples of prime numbers: 5, 43, 97

\pause\transdissolve\bigskip

There are infinitely many primes

\pause\transdissolve\bigskip

A natural number greater than 1 that is not prime is called a composite number

\pause\transdissolve\bigskip

Examples of composite numbers: 8, 46, 99

\pause\transdissolve\bigskip

Prime factorization is the decomposition of a composite number into a product of primes

\pause\transdissolve\bigskip

For example, the prime factorization of 18 is $2 \times 3 \times 3$
\end{frame}

\section{Operations on Numbers}
\begin{frame}[fragile]
\pause\transdissolve

Addition ($x + y$)

\pause\transdissolve\bigskip

Subtraction ($x - y$)

\pause\transdissolve\bigskip

Multiplication ($x \times y$)

\pause\transdissolve\bigskip

Division ($x \div y$)

\pause\transdissolve\bigskip

Exponentiation ($x^y$)

\pause\transdissolve\bigskip

Order of operations: (P)arenthesis, (E)xponents, (M)ultiplication, (D)ivision, (A)ddition, (S)ubtraction

\pause\transdissolve\bigskip

For example, $(5 - 3) \times 2^3 + 24 \div 4 = 262$

\pause\transdissolve\bigskip

If $x$ and $d$ ($\neq 0$) are integers, then there exist unique integers $q$ (quotient) and $r$ (remainder), such that $x = q \times d + r$ and $0 \leq r \leq |d|$

\pause\transdissolve\bigskip

For example, $43 = 8 \times 5 + 3$
\end{frame}

\begin{frame}[fragile]
\pause\transdissolve

Laws of exponents
\visible<2->{
\tcbox[enhanced,drop shadow southwest,sharp corners,size=fbox,colback=white]{
\begin{tabular}{ll}
\textbf{Law} & \textbf{Example} \\ \hline \\
$x^1=x$ & $6^1=6$ \\
$x^0=1$ & $7^0=1$ \\
$x^mx^n=x^{m+n}$ & $x^2x^3=x^5$ \\ 
$x^m/x^n=x^{m-n}$ & $x^6/x^2=x^4$ \\ 
$(x^m)^n=x^{mn}$ & $(x^2)^3=x^6$ \\
$(xy)^n=x^ny^n$ & $(xy)^3=x^3y^3$ \\
$(x/y)^n=x^n/y^n$ & $(x/y)^2=x^2/y^2$ \\
$x^{-n}=1/x^n$ & $x^{-3}=1/x^3$ \\
$x^{m/n}=\sqrt[n]{x^m}=(\sqrt[n]{x})^m$ & $x^{2/3}=\sqrt[3]{x^2}=(\sqrt[3]{x})^2$ \\
\end{tabular}
}
}
\end{frame}

\section{Functions}
\begin{frame}[fragile]
\pause\transdissolve

\end{frame}

\section{Basic Probability}
\begin{frame}[fragile]
\pause\transdissolve

\end{frame}

\section{Basic Statistics}
\begin{frame}[fragile]
\pause\transdissolve


\end{frame}
\end{document}

\documentclass[8pt,a4paper,compress]{beamer}

\usepackage{/home/siyer/lib/slides}

\title{Mathematical Preliminaries}
\date{}

\newlength{\myMheight}
\settoheight{\myMheight}{M}

\begin{document}
\begin{frame}
\vfill
\titlepage
\end{frame}

\section{Numbers}
\begin{frame}[fragile]
\pause\transdissolve

An integer is a number that can be written without a fractional component

\pause\transdissolve\bigskip

Examples of integers: -128, 0, 42

\pause\transdissolve\bigskip

The set of all integers is denoted as $\mathbb{Z}$

\pause\transdissolve\bigskip

The set $\mathbb{Z}$ includes zero (0), the natural numbers $\mathbb{N}$ ($1, 2, 3, \dots$), and their additive inverses ($-1, -2, -3, \dots$)
\end{frame}

\begin{frame}[fragile]
\pause\transdissolve

A rational number is a number that can be expressed as the quotient or fraction $p/q$ of two integers, a numerator $p$ and a non-zero denominator $q$

\pause\transdissolve\bigskip

Examples of rational numbers: 42, 2.5, 1.333...

\pause\transdissolve\bigskip

The set of all rational numbers is denoted as $\mathbb{Q}$
\end{frame}

\begin{frame}[fragile]
\pause\transdissolve

A real number is a value of a continuous quantity such as distance or time

\pause\transdissolve\bigskip

Examples of real numbers: -7, 0, 42, 1.333..., $\sqrt{2}$ (1.41421356...), $\pi$ (3.14159265...)

\pause\transdissolve\bigskip

The set of all real numbers is denoted as $\mathbb{R}$

\pause\transdissolve\bigskip

Hierarchy of number systems: $\mathbb{N} \subset \mathbb{Z} \subset \mathbb{Q} \subset \mathbb{R}$

\pause\transdissolve\bigskip

A real number that is not rational is called an irrational number

\pause\transdissolve\bigskip

Examples of irrational numbers: $\sqrt{2}$ (1.41421356...), $\pi$ (3.14159265...)
\end{frame}

\begin{frame}[fragile]
\pause\transdissolve

A prime number is a natural number greater than 1 that cannot be formed by multiplying two smaller natural numbers

\pause\transdissolve\bigskip

Examples of prime numbers: 5, 43, 97

\pause\transdissolve\bigskip

There are infinitely many primes

\pause\transdissolve\bigskip

A natural number greater than 1 that is not prime is called a composite number

\pause\transdissolve\bigskip

Examples of composite numbers: 8, 46, 99
\end{frame}

\section{Arithmetic Operations}
\begin{frame}[fragile]
\pause\transdissolve

\end{frame}

\section{Functions}
\begin{frame}[fragile]
\pause\transdissolve

\end{frame}

\section{Basic Probability and Statistics}
\begin{frame}[fragile]
\pause\transdissolve

\end{frame}
\end{document}

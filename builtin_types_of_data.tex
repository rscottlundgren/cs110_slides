\documentclass[8pt,a4paper,compress,handout]{beamer}

\usepackage{/home/siyer/lib/slides}

\title{Built-in Types of Data}
\date{}

\begin{document}
\begin{frame}
\vfill
\titlepage
\end{frame}

\begin{frame}
\frametitle{Outline}
\tableofcontents
\end{frame}

\section{Types}
\begin{frame}[fragile]
A \emph{data type} is set of values and a set of operations defined on those values.

\bigskip

Python's built-in data types include \lstinline{int} (for integers), \lstinline{float} (for floating-point numbers), \lstinline{str} (for sequences of characters), and \lstinline{bool} (for true-false values).

\bigskip

Python also allows you to compose your own data types.
\end{frame}

\section{Definitions}
\begin{frame}[fragile]
A \emph{literal} is a Python-code representation of a data-type value. For example, \lstinline{1234} and \lstinline{99} are \lstinline{int} literals; \lstinline{3.14159} and \lstinline{2.71828} are \lstinline{float} literals; \lstinline{True} and \lstinline{False} are \lstinline{bool} literals; \lstinline{'Hello, World'} is a \lstinline{str} literal.

\bigskip

An \emph{operator} is a Python-code representation of a data-type operation. For example, \lstinline{+} and \lstinline{*} represent addition and multiplication for integers and floating-point numbers; \lstinline{and}, \lstinline{or}, and \lstinline{not} represent boolean operations.

\bigskip

An \emph{identifier} is a Python-code representation of a name. Each identifier is a sequence of letters, digits, and underscores, the first of which is not a digit. For example, \lstinline{abc}, \lstinline{Ab_}, \lstinline{abc123}, and \lstinline{a_b} are valid identifiers, but \lstinline{Ab*}, \lstinline{1abc}, and \lstinline{a+b} are not.

\bigskip

Certain \emph{keywords}, such as \lstinline{and}, \lstinline{import}, \lstinline{in}, \lstinline{def}, \lstinline{while}, \lstinline{from}, and \lstinline{lambda}, are reserved, and you cannot use them as identifiers. Other names, such as \lstinline{int}, \lstinline{sum}, \lstinline{min}, \lstinline{max}, \lstinline{len}, \lstinline{id}, \lstinline{file}, and \lstinline{input}, have special meaning, so it is best not to use them, either.
\end{frame}

\begin{frame}[fragile]
A \emph{variable} is a name associated with a data-type value. For example, the variable \lstinline{total} might represent the running total of sequence of numbers.

\bigskip

A \emph{constant variable} describes a variable whose associated a data-type value does not change during the execution of a program. For example, the variables \lstinline{SPEED_OF_LIGHT} and \lstinline{DARK_RED} might represent the speed of light and a shade of red, respectively.

\bigskip

An \emph{expression} is a combination of literals, variables, and operators that Python evaluates to produce a value. For example, \lstinline{4 * (x - 3)} is an expression.

\bigskip

Python has a natural and well-defined set of \emph{precedence rules} that fully specify the order in which the operators are applied in an expression. 
\begin{itemize}
\item For arithmetic operations, multiplication and division are performed before addition and subtraction. 

\item When arithmetic operations have the same precedence, they are \emph{left associative}, with the exception of the exponentiation operator \lstinline{**}, which is \emph{right associative}.

\item You can use parentheses to override precedence rules. 
\end{itemize} 
\end{frame}

\begin{frame}[fragile]
We use an \emph{assignment statement} to define a variable and associate it with a data-type value.
\begin{lstlisting}[language={}]
<variable> = <value>
\end{lstlisting}
For exampe, the statement 
\begin{lstlisting}[language={}]
a = 1234
\end{lstlisting}
defines an identifier \lstinline{a} to be a new variable and associates it with the integer data-type value \lstinline{1234}.

\bigskip

All data values in Python are represented by \emph{objects}, each characterized by its \emph{identity} (or \emph{memory address}), \emph{type}, and \emph{value}.

\bigskip

An \emph{object reference} is a concrete representation of the object's identity.

\bigskip


\end{frame}

\section{Strings}
\begin{frame}[fragile]
blah
\end{frame}

\begin{frame}[fragile]
\begin{framed}
\tiny ruler.py: The ruler function $R(n)$ is the exponent of the largest power of 2 which 
divides $2n$. The $i$th row in the output lists the values of $R(n)$ for $n=1,2,
\dots,2^i-1$.
\end{framed}

\begin{lstlisting}[language=Python]
import stdio

ruler1 = '1'
ruler2 = ruler1 + ' 2 ' + ruler1
ruler3 = ruler2 + ' 3 ' + ruler2
ruler4 = ruler3 + ' 4 ' + ruler3
stdio.writeln(ruler1)
stdio.writeln(ruler2)
stdio.writeln(ruler3)
stdio.writeln(ruler4)
\end{lstlisting}

\begin{lstlisting}[language={}]
$ python ruler.py 
1
1 2 1
1 2 1 3 1 2 1
1 2 1 3 1 2 1 4 1 2 1 3 1 2 1
\end{lstlisting}
\end{frame}

\begin{frame}[fragile]
\begin{framed}
\tiny intops.py: Accept two integers $a$ and $b$ as command-line arguments, perform integer operations on them, and write the results to standard output.
\end{framed}

\begin{lstlisting}[language=Python]
import stdio
import sys

a = int(sys.argv[1])
b = int(sys.argv[2])
sum  = a +  b
diff = a -  b
prod = a *  b
quot = a // b
rem  = a %  b
exp  = a ** b
stdio.writeln(str(a) + ' +  ' + str(b) + ' = ' + str(sum))
stdio.writeln(str(a) + ' -  ' + str(b) + ' = ' + str(diff))
stdio.writeln(str(a) + ' *  ' + str(b) + ' = ' + str(prod))
stdio.writeln(str(a) + ' // ' + str(b) + ' = ' + str(quot))
stdio.writeln(str(a) + ' %  ' + str(b) + ' = ' + str(rem))
stdio.writeln(str(a) + ' ** ' + str(b) + ' = ' + str(exp))
\end{lstlisting}

\begin{lstlisting}[language={}]
$ python intops.py 1234 5
1234 +  5 = 1239
1234 -  5 = 1229
1234 *  5 = 6170
1234 // 5 = 246
1234 %  5 = 4
1234 ** 5 = 2861381721051424
\end{lstlisting}
\end{frame}

\begin{frame}[fragile]
\begin{framed}
\tiny floatops.py: Accept two floats $a$ and $b$ as command-line arguments, perform floating-point operations on them, and write the results to standard output.
\end{framed}

\begin{lstlisting}[language=Python]
import stdio
import sys

a = float(sys.argv[1])
b = float(sys.argv[2])
sum  = a + b
diff = a - b
prod = a * b
quot = a / b
exp  = a ** b
stdio.writeln(str(a) + ' +  ' + str(b) + ' = ' + str(sum))
stdio.writeln(str(a) + ' -  ' + str(b) + ' = ' + str(diff))
stdio.writeln(str(a) + ' *  ' + str(b) + ' = ' + str(prod))
stdio.writeln(str(a) + ' /  ' + str(b) + ' = ' + str(quot))
stdio.writeln(str(a) + ' ** ' + str(b) + ' = ' + str(exp))
\end{lstlisting}

\begin{lstlisting}[language={}]
$ python floatops.py 123.456 78.9
123.456 +  78.9 = 202.356
123.456 -  78.9 = 44.556
123.456 *  78.9 = 9740.6784
123.456 /  78.9 = 1.5647148289
123.456 ** 78.9 = 1.04788279167e+165
\end{lstlisting}
\end{frame}

\begin{frame}[fragile]
\begin{framed}
\tiny quadratic.py: Accept floats $b$ and $c$ as command-line arguments, compute the the roots of the polynomial $x^2 + bx + c$ using the quadratic formula $x=(-b\pm \sqrt{b^2-4c})/2$, and write the roots to standard output.
\end{framed}

\begin{lstlisting}[language=Python]
import math
import stdio
import sys

b = float(sys.argv[1])
c = float(sys.argv[2])
discriminant = b * b - 4.0 * c
d = math.sqrt(discriminant)
stdio.writeln((-b + d) / 2.0)
stdio.writeln((-b - d) / 2.0)
\end{lstlisting}

\begin{lstlisting}[language={}]
$ python quadratic.py -3.0 2.0
2.0
1.0
$ python quadratic.py -1.0 -1.0
1.61803398875
-0.61803398875
$ python quadratic.py 1.0 1.0
Traceback (most recent call last):
  File "quadratic.py", line 17, in <module>
    d = math.sqrt(discriminant)
ValueError: math domain error
\end{lstlisting}
\end{frame}

\begin{frame}[fragile]
\begin{framed}
\tiny leapyear.py: Accept an integer $year$ as command-line argument, and write \lstinline{True} to standard output if $year$ is a leap year and \lstinline{False} otherwise. 
\end{framed}

\begin{lstlisting}[language=Python]
import stdio
import sys

year = int(sys.argv[1])
isLeapYear = (year % 4 == 0)
isLeapYear = isLeapYear and (year % 100 != 0)
isLeapYear = isLeapYear or  (year % 400 == 0)
stdio.writeln(isLeapYear)
\end{lstlisting}

\begin{lstlisting}[language={}]
$ python leapyear.py 2016
True
$ python leapyear.py 1900
False
$ python leapyear.py 2000
True
\end{lstlisting}
\end{frame}

\end{document}

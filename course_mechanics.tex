\documentclass[8pt,a4paper,compress]{beamer}

\usepackage{/home/siyer/lib/slides}

\title{Course Mechanics}
\date{}

\begin{document}
\begin{frame}
\vfill
\titlepage
\end{frame}

\begin{frame}
\frametitle{Outline}
\tableofcontents
\end{frame}

\section{Course Logistics}
\begin{frame}[fragile]
Website: \href{http://www.swamiiyer.net/cs110}{http://www.swamiiyer.net/cs110}

\bigskip

Goal: proficiency in the design and implementation of (Python) programs of significant size and complexity

\bigskip

Prerequisites: Math 140 credits or placement; or Math 130 as corequisite; or permission of the department

\bigskip

Instructor: Swami Iyer

\bigskip

Twice a week classes

\bigskip
Weekly discussion sessions

\bigskip

Weekly supplemental instruction sessions (optional)

\bigskip

Tutoring

\bigskip

Text: 
\begin{itemize}
\item CS for All (free online text) by Christine Alvarado et al
\item Introduction to Programming in Python: An Interdisciplinary Approach by Robert Sedgewick et al
\end{itemize}
\end{frame}

\begin{frame}[fragile]
Grading:
\begin{itemize}
\item Homework assignments (best 7 of 10): 7\%
\item Programming assignments (best 6 of 7): 30\%
\item Exams (best 3 of 4): 60\%
\item Attendance (class and discussion session): 3\%
\end{itemize}

\bigskip

CS account

\bigskip

Piazza Q\&A platform

\bigskip

Policies
\begin{itemize}
\item Classroom etiquette
\item Piazza etiquette
\item Late assignments
\item Collaboration
\item Code of conduct
\item Accommodations for students with disabilities
\end{itemize}
\end{frame}

\begin{frame}[fragile]
Tips to succeed

\bigskip

Other items on the course website:
\begin{itemize}
\item Announcements (landing page)
\item Calendar
\item Slides 
\item Assignments
\item Resources
\end{itemize}

\bigskip

Things to do immediately:
\begin{itemize}
\item Sign up for the course on Piazza
\item Setup the programming environment
\item Apply for a CS account
\end{itemize}
\end{frame}

\section{Course Overview}

\begin{frame}[fragile]
Introduction
\begin{itemize}
\item What is Computer Science?
\item PicoBot
\end{itemize}

\smallskip

Computer Organization
\begin{itemize}
\item Building a Computer
\item The Harvey Mudd Miniature Machine (HMMM)
\end{itemize}

\smallskip

Imperative Programming
\begin{itemize}
\item Your First Program
\item Built-in Types of Data
\item Control Flow
\item Collections
\item Input and Output
\item Case Study: What Makes Google Different? (PageRank Algorithm)
\end{itemize}
\end{frame}

\begin{frame}[fragile]
Functional Programming
\begin{itemize}
\item Defining Functions
\item Modules and Clients
\item Recursion
\item Case Study: Fermi's Paradox (Percolation Problem)
\end{itemize}

\bigskip

Object-oriented Programming
\begin{itemize}
\item Using Data Types
\item Creating Data Types
\item Designing Data Types
\item Case Study: The Music of the Spheres ($N$-body Problem)
\end{itemize}

\bigskip

Algorithms and Data Structures
\begin{itemize}
\item Searching and Sorting
\item Stacks and Queues
\item Case Study: Six Degrees of Separation (Small-world Problem)
\end{itemize}

\bigskip

Problem ``Hardness''
\end{frame}
\end{document}

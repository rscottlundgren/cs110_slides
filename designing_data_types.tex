\documentclass[8pt,a4paper,compress,handout]{beamer}

\usepackage{/home/siyer/lib/slides}

\title{Designing Data Types}
\date{}

\begin{document}
\begin{frame}
\vfill
\titlepage
\end{frame}

\begin{frame}
\frametitle{Outline}
\tableofcontents
\end{frame}

\section{APIs}
\begin{frame}[fragile]
Precisely specifying a data type using an API improves design because it leads to client code that can clearly express its computation. 

\bigskip

By using APIs to separate clients from implementations, we reap the benefits of standard interfaces for every program that we compose.

\bigskip

We should avoid APIs that are too hard to implement, too narrow, too wide, too general, or too specific.
\end{frame}

\section{Encapsulation}
\begin{frame}[fragile]
The process of separating clients from implementations by hiding information is known as \emph{encapsulation}.

\bigskip

Encapsulation allows one implementation of an API to be substituted for another.

\bigskip

Encapsulation also helps programmers ensure that their code operates as intended.

\bigskip

Python does not enforce encapsulation; instead, through a naming convention, clients are informed that they should not directly access the instance variable, method, or function thus named.

\bigskip

\emph{Modular programming} is insisting on the API being the only point of dependence between client and implementation.
\end{frame}

\begin{frame}[fragile]
A data type \lstinline{Complex} for complex numbers:
\begin{center}
\begin{tabular}{cc}
method & description \\ \hline
\lstinline$Complex(x, y)$ & a new complex object $c$ with value $x + yi$ \\
\lstinline$c.re()$ & real part of $c$ \\
\lstinline$c.im()$ & imaginary part of $c$ \\
\lstinline$c.conjugate()$ & conjugate of $c$ \\
\lstinline$c + d$ & sum of $c$ and $d$ \\
\lstinline$c * d$ & product of $c$ and $d$ \\
\lstinline$abs(c)$ & magnitute of $c$ \\
\lstinline$str(c)$ & \lstinline$x + yi$ (string representation of $c$)
\end{tabular} 
\end{center}
\end{frame}

\begin{frame}[fragile]
\begin{framed}
\tiny \lstinline{complexpolar.py}: \lstinline{Complex} data type redux. 
\end{framed}

\begin{lstlisting}[language=Python]
import math
import stdio

class Complex:
    def __init__(self, re = 0.0, im = 0.0):
        self._r = math.sqrt(re * re + im * im)
        self._theta = math.atan2(im, re)

    def re(self):
        return self._r * math.cos(self._theta)

    def im(self):
        return self._r * math.sin(self._theta)

    def conjugate(self):
        return Complex(self.re(), -self.im())

    def __add__(self, other):
        re = self.re() + other.re()
        im = self.im() + other.im()
        return Complex(re, im)

    def __mul__(self, other):
        c = Complex(0, 0)
        c._r = self._r * other._r
        c._theta = self._theta + other._theta
        return c
\end{lstlisting}
\end{frame}

\begin{frame}[fragile]
\begin{lstlisting}[language=Python]
    def __abs__(self):
        return self._r

    def __str__(self):
        return str(self.re()) + ' + ' + str(self.im()) + 'i'

def main():
    z0 = Complex(1.0, 1.0)
    z = z0
    z = z * z + z0
    z = z * z + z0
    stdio.writeln(z)

if __name__ == '__main__':
    main()
\end{lstlisting}

\begin{lstlisting}[language={}]
$ python complexpolar.py 
-7.0 + 7.0i
\end{lstlisting}
\end{frame}

\begin{frame}[fragile]
\begin{framed}
\tiny \lstinline{counter.py}: Defines a data type \lstinline{Counter}.
\end{framed}

\begin{lstlisting}[language=Python]
import stdio
import stdrandom
import sys

class Counter:
    def __init__(self, id, maxCount):
        self._name = id
        self._maxCount = maxCount
        self._count = 0

    def increment(self):
        if self._count < self._maxCount:
            self._count += 1

    def value(self):
        return self._count

    def __str__(self):
        return self._name + ': ' + str(self._count)

    def __eq__(self, other):
        return self._count == other._count

    def __ne__(self, other):
        return self._count != other._count

    def __lt__(self, other):
        return self._count < other._count
\end{lstlisting}
\end{frame}

\begin{frame}[fragile]
\begin{lstlisting}[language=Python]
    def __gt__(self, other):
        return self._count > other._count

    def __le__(self, other):
        return self._count <= other._count

    def __ge__(self, other):
        return self._count >= other._count

def main():
    n = int(sys.argv[1])
    p = float(sys.argv[2])
    heads = Counter('Heads', n)
    tails = Counter('Tails', n)
    for i in range(n):
        if stdrandom.bernoulli(p):
            heads.increment()
        else:
            tails.increment()
    stdio.writeln(heads)
    stdio.writeln(tails)

if __name__ == '__main__':
    main()
\end{lstlisting}

\begin{lstlisting}[language={}]
$ python counter.py 1000 .5
Heads: 483
Tails: 517
$ python counter.py 1000 .5
Heads: 503
Tails: 497
$ python counter.py 1000 .3
Heads: 280
Tails: 720
\end{lstlisting}
\end{frame}

\section{Immutability}
\begin{frame}[fragile]
blah
\end{frame}

\section{Polymorphism}
\begin{frame}[fragile]
blah
\end{frame}

\section{Overloading}
\begin{frame}[fragile]
blah
\end{frame}

\section{Functions are Objects}
\begin{frame}[fragile]
blah
\end{frame}

\section{Inheritance}
\begin{frame}[fragile]
blah
\end{frame}

\section{Application: Data Mining}
\begin{frame}[fragile]
\begin{framed}
\tiny \lstinline{vector.py}: Defines a data type \lstinline{Vector}.
\end{framed}

\begin{lstlisting}[language=Python]
import math
import stdarray
import stdio

class Vector:
    def __init__(self, a):
        self._coords = a[:]
        self._n = len(a)

    def __getitem__(self, i):
        return self._coords[i]

    def __add__(self, other):
        result = stdarray.create1D(self._n, 0)
        for i in range(self._n):
            result[i] = self._coords[i] + other._coords[i]
        return Vector(result)

    def __sub__(self, other):
        result = stdarray.create1D(self._n, 0)
        for i in range(self._n):
            result[i] = self._coords[i] - other._coords[i]
        return Vector(result)

    def scale(self, alpha):
        result = stdarray.create1D(self._n, 0)
        for i in range(self._n):
            result[i] = alpha * self._coords[i]
        return Vector(result)
\end{lstlisting}
\end{frame}

\begin{frame}[fragile]
\begin{lstlisting}[language=Python]
    def dot(self, other):
        result = 0
        for i in range(self._n):
            result += self._coords[i] * other._coords[i]
        return result

    def __abs__(self):
        return math.sqrt(self.dot(self))

    def direction(self):
        return self.scale(1.0 / abs(self))

    def __str__(self):
        return str(self._coords)
        
    def __len__(self):
        return self._n
        
def main():
    xCoords = [1.0, 2.0, 3.0, 4.0]
    yCoords = [5.0, 2.0, 4.0, 1.0]
    x = Vector(xCoords)
    y = Vector(yCoords)
    stdio.writeln('x        = ' + str(x))
    stdio.writeln('y        = ' + str(y))
    stdio.writeln('x + y    = ' + str(x + y))
    stdio.writeln('10x      = ' + str(x.scale(10.0)))
    stdio.writeln('|x|      = ' + str(abs(x)))
    stdio.writeln('<x, y>   = ' + str(x.dot(y)))
    stdio.writeln('|x - y|  = ' + str(abs(x - y)))

if __name__ == '__main__':
    main()
\end{lstlisting}

\begin{lstlisting}[language={}]

\end{lstlisting}
\end{frame}

\begin{frame}[fragile]
\begin{framed}
\tiny \lstinline{sketch.py}: Defines a data type \lstinline{Sketch}. 
\end{framed}

\begin{lstlisting}[language=Python]
import stdarray
import stdio
import sys
from vector import Vector

class Sketch:
    def __init__(self, text, k, d):
        freq = stdarray.create1D(d, 0)
        for i in range(len(text) - k):
            kgram = text[i:i + k]
            h = hash(kgram)
            freq[h % d] += 1
        vector = Vector(freq)
        self._sketch = vector.direction()

    def similarTo(self, other):
        return self._sketch.dot(other._sketch)

    def __str__(self):
        return str(self._sketch)

def main():
    text = stdio.readAll()
    k = int(sys.argv[1])
    d = int(sys.argv[2])
    sketch = Sketch(text, k, d)
    stdio.writeln(sketch)

if __name__ == '__main__':
    main()
\end{lstlisting}
\end{frame}

\begin{frame}[fragile]
\begin{lstlisting}[language={}]
$ more genome20.txt
ATAGATGCATAGCGCATAGC
$ python sketch.py 2 16 < genome20.txt 
[0.37210420376762543, 0.37210420376762543, 0.49613893835683387, 0.0, 
0.12403473458920847, 0.0, 0.0, 0.0, 0.0, 0.0, 0.24806946917841693, 0.0, 
0.12403473458920847, 0.6201736729460423, 0.0, 0.0]
\end{lstlisting}
\end{frame}

\begin{frame}[fragile]
\begin{framed}
\tiny \lstinline{comparedocuments.py}: Accepts integers \carg{k} and \carg{d} as command-line arguments, reads a document list from standard input, computes \carg{d}-dimensional profiles based on \carg{k}-gram frequencies for all the documents, and writes a matrix of similarity measures between all pairs of documents.
\end{framed}

\begin{lstlisting}[language=Python]
import stdarray
import stdio
import sys
from instream import InStream
from sketch import Sketch

def main():
    k = int(sys.argv[1])
    d = int(sys.argv[2])
    filenames = stdio.readAllStrings()
    sketches = stdarray.create1D(len(filenames))
    for i in range(len(filenames)):
        text = InStream(filenames[i]).readAll()
        sketches[i] = Sketch(text, k, d)
    stdio.write('    ')
    for i in range(len(filenames)):
        stdio.writef('%8.4s', filenames[i])
    stdio.writeln()
    for i in range(len(filenames)):
        stdio.writef('%.4s', filenames[i])
        for j in range(len(filenames)):
            stdio.writef('%8.2f', sketches[i].similarTo(sketches[j]))
        stdio.writeln()
    
if __name__ == '__main__':
    main()
\end{lstlisting}
\end{frame}

\begin{frame}[fragile]
\begin{lstlisting}[language={}]
$ more documents.txt
constitution.txt
tomsawyer.txt
huckfinn.txt
prejudice.txt
vector.py
djia.csv
amazon.html
actg.txt
$ python comparedocuments.py 5 10000 < documents.txt
        cons    toms    huck    prej    vect    djia    amaz    actg
cons    1.00    0.67    0.61    0.64    0.10    0.18    0.19    0.12
toms    0.67    1.00    0.93    0.87    0.08    0.23    0.19    0.15
huck    0.61    0.93    1.00    0.81    0.06    0.21    0.15    0.14
prej    0.64    0.87    0.81    1.00    0.07    0.25    0.19    0.16
vect    0.10    0.08    0.06    0.07    1.00    0.03    0.17    0.01
djia    0.18    0.23    0.21    0.25    0.03    1.00    0.13    0.12
amaz    0.19    0.19    0.15    0.19    0.17    0.13    1.00    0.09
actg    0.12    0.15    0.14    0.16    0.01    0.12    0.09    1.00
\end{lstlisting}
\end{frame}

\section{Design by Contract}
\begin{frame}[fragile]
When using the design-by-contract model, the designer of a data type expresses:
\begin{itemize}
\item a \emph{precondition} - the condition that the client promises to satisfy when calling a method;
\item a \emph{postcondition} - the condition that the implementation promises to achieve when returning from a method;
\item \emph{invariants} - any condition that the implementation promises to satisfy while the method is executing; and 
\item \emph{side effects} - any other change in state that the method could cause.
\end{itemize}

\bigskip

\emph{Exceptions} and \emph{assertions} are Python language mechanisms that enable us to test these conditions.
\end{frame}

\begin{frame}[fragile]
An \emph{exception} is a disruptive event that occurs while a program is running, often to signal an error. 

\bigskip

The action taken is known as \emph{raising an exception} (or \emph{error}).

\bigskip

We can raise our own exceptions as follows:
\begin{lstlisting}[language=Python]
raise Exception('Error message here.')
\end{lstlisting} 

\bigskip

For example, in \lstinline{vector.py}, we can raise an exception in \lstinline{__add__()} if the two \lstinline{Vectors} to be added have different dimensions. 
\begin{lstlisting}[language=Python]
if len(self) != len(other):
    raise Exception('vectors have different dimensions')
\end{lstlisting} 

\bigskip

We handle exceptions using a try-except block. For example
\begin{lstlisting}[language=Python]
(x, y) = (5, 0)
try:
    z = x / y
except ZeroDivisionError as e:
    z = e
stdio.writeln(z)
\end{lstlisting} 
\end{frame}

\begin{frame}[fragile]
An \emph{assertion} is a boolean expression that we affirm is \lstinline{True}, and if it is \lstinline{False}, the program will raise an \lstinline{AssertionError} at run time.

\bigskip
For example, in \lstinline{counter.py}, we might check that the counter is never negative by adding the following assertion as the last statement in \lstinline{increment()}:
\begin{lstlisting}[language=Python]
assert self.__count >= 0
\end{lstlisting} 

We can also put an optional message, such as
\begin{lstlisting}[language=Python]
assert self.__count >= 0, 'Negative count detected'
\end{lstlisting} 
\end{frame}
\end{document}
